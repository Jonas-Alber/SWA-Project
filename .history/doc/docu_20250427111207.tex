\documentclass{article}

% Optional packages
\usepackage[utf8]{inputenc} % Allows UTF-8 input
\usepackage[T1]{fontenc}    % Selects font encodings
\usepackage{amsmath}        % For mathematical formulas
\usepackage{graphicx}       % To include images
\usepackage[ngerman]{babel} % German language support

% Document information
\title{Beispiel Dokument}
\author{Dein Name}
\date{\today}

\begin{document}

\maketitle % Displays title, author, and date

\section{Einleitung}
Dies ist ein einfaches Beispiel für ein LaTeX-Dokument.
Hier können Sie Ihren Text einfügen.

\section{Einsatzzweck}
Dieser Abschnitt enthält den Hauptinhalt des Dokuments.
Sie können mathematische Formeln wie $E=mc^2$ einfügen oder Bilder:
% \includegraphics[width=0.5\textwidth]{beispiel-bild.png} % Uncomment to include an image

\section{Vor- und Nachteile}
Dieser Abschnitt enthält den Hauptinhalt des Dokuments.
Sie können mathematische Formeln wie $E=mc^2$ einfügen oder Bilder:
% \includegraphics[width=0.5\textwidth]{beispiel-bild.png} % Uncomment to include an image

\section{Praktische Umsetzung}
Dieser Abschnitt enthält den Hauptinhalt des Dokuments.
Sie können mathematische Formeln wie $E=mc^2$ einfügen oder Bilder:
% \includegraphics[width=0.5\textwidth]{beispiel-bild.png} % Uncomment to include an image

\subsection{Gesamtarchitektur}
Weitere Details können in Unterabschnitten hinzugefügt werden.

\section{Schlussfolgerung}
Zusammenfassung des Dokuments.

\end{document}